\section{Requerimientos de ChiVO}

Los requerimientos se clasificarán en:
\begin{itemize}
	\item Necesidad: Esencial, deseable, opcional.
	\item Prioridad temporal: Alta, media, baja.
\end{itemize}

\begin{enumerate}
	\item \textbf{Buscar por coordenadas o región del cielo [Necesidad:
Esencial | Prioridad temporal: Alta]:}

Se podrán realizar búsquedas de posición mediante coordenadas y radio angular
(cónicas) o por región del cielo.

Los parámetros de las coordenadas pueden ser en distintos sistemas como
ecuatorial, eclíptico, galáctico o supergaláctico. Los parámetros ingresados se
convertirán al sistema de la fuente de datos, para así poder realizar las
búsquedas, IVOA utiliza los sistemas de coordenadas ICRS y ecuatoriales J2000.

En un principio, el sistema ofrecerá servicio de búsqueda por coordenadas
cónicas y más adelante, en caso de ser vía un portal web, se ofrecerá búsqueda
por región de cielo.

El sistema también deberá permitir buscar simultáneamente un listado de
coordenadas.

	\item \textbf{Buscar por nombre o tipo de objeto [Necesidad: Esencial |
Prioridad temporal: Alta]:}

El sistema deberá permitir buscar por nombres de objetos que se encuentren
definidos en Sesame, del Centre de Données astronomiques de Strasbourg (CDS).

Por otro lado, la búsqueda por tipo o subtipo de objeto, tales como estrellas
en formación, estrellas nebulosas planetarias, supernovas, galaxias, cometas,
entre otros, permitirá al usuario encontrar datos relacionados con una
problemática en especial. En un principio, el sistema realizará estas búsquedas
acorde a la información presente en los catálogos.

A futuro, el sistema deberá permitir minería de datos para detección de tipos
de objetos similares, esto es posible debido a que existen clasificaciones
discretas que permiten clasificar los objetos que se encuentran en las
observaciones.

Este tipo de búsquedas, se transforman en búsquedas por coordenadas, ya que al
buscar por un nombre, por ejemplo, Sesame responde con la correspondiente
ubicación del objeto en coordenadas. Y luego de ello se procede a realizar la
búsqueda por coordenadas correspondiente.

El resultado de la búsqueda deberá facilitar la obtención de datos para ser
analizados como secuencias de tiempo.

	\item \textbf{Buscar por metadatos espectrales (frecuencia y
resolución) [Necesidad: Esencial | Prioridad temporal: Alta]:}

Se podrán realizar búsquedas por metadatos espectrales, lo cual consiste en
búsqueda por banda o rango de frecuencia, búsquedas por líneas espectrales y
corrimiento al rojo o búsquedas por resolución espectral. Específicamente, hay
dos enfoques:
\begin{itemize}
	\item Galáctico: Por frecuencia en reposo y velocidad radial.
	\item Extragaláctico: Por frecuencia en reposo y corrimiento al rojo.
\end{itemize}

La Frecuencia en reposo incluye búsquedas por molécula, transición de molécula
(vibracional, rotacional o electrónica) o frecuencia de línea espectral.

	\item \textbf{Buscar por metadatos espaciales (resolución angular y
campos de visión) [Necesidad: Esencial | Prioridad temporal: Alta]:}

Se podrán realizar búsquedas espaciales en base a parámetros relacionados con
rangos de  resolución angular y campos de visión y siempre en base a
coordenadas.

Los parámetros de las coordenadas pueden ser en distintos sistemas como
ecuatorial, eclíptico, galáctico o supergaláctico. Los parámetros ingresados se
convertirán al sistema de la fuente de datos, para así poder realizar las
búsquedas.

Además, se podrá especificar parámetros relacionados con la forma de las
observaciones (rectangulares o redondas).

	\item \textbf{Buscar por metadatos temporales [Necesidad: Esencial | Prioridad temporal: Baja]:}
Se podrán realizar búsquedas por metadatos temporales que pueden ser
clasificadas en dos tipos de búsquedas: 
\begin{itemize} 
\item Cuando fue realizada la observación, incluyendo cuantas veces se observó
un objeto y/o el intervalo entre observaciones.  
\item Nivel de ruido, duración de la observación o tiempo de integración. Dado
un ruido, se necesita un tiempo de integración que depende de la frecuencia
observada y del clima.  
\end{itemize}

	\item \textbf{Buscar por polarización [Necesidad: Esencial | Prioridad
temporal: Media]:}

Cada imagen se puede dividir en cuatro parámetros llamados los parámetros de
Stokes o en dos: izquierda y derecha. En radio astronomía no suele hacerse
debido a que requiere una alta precisión del instrumento, en el caso de ALMA se
requiere que esté lista la calibración.

El sistema debe permitir buscar si existe o no polarización en alguno de los
parámetros de Stokes: I, Q, U o V.

	\item \textbf{Cruzamiento de información [Necesidad: Esencial |
Prioridad temporal: Alta]:}

La búsqueda cruzada debe permitir al usuario realizar las búsquedas mencionadas
anteriormente en múltiples fuentes de datos distribuidos globalmente, sin
importar su tipo. Lo que permitirá obtener todos los datos existentes sobre un
objeto o área espacial y así evitar realizar observaciones innecesarias debido
al descubrimiento de observaciones existentes. Los tipos de fuentes pueden ser
Sesame, ALMA u Observatorios Virtuales, que cumplan con los estándares de IVOA.

La búsqueda de un mismo objeto en distintas fuentes de información, debido a
que en cada fuente el instrumento tiene un margen de error en cuanto a la
posición del objeto, debe ser capaz de realizar una intersección entre los
radios de margen de error de las distintas fuentes para identificar al objeto
en una búsqueda cruzada.

	\item \textbf{Simulaciones [Necesidad: Deseable | Prioridad temporal:
Baja]:}

Es a veces necesario realizar comparaciones con observaciones obtenidas a
través de simulaciones, así cómo es costoso  realizar una observación dos
veces, para una simulación puede ser aún más, debido a que dependiendo de la
magnitud, hay simulaciones que requieren una gran capacidad de cómputo.

	\item \textbf{Servicios Bibliográficos [Necesidad: Deseable | Prioridad
temporal: Baja]:}

Las herramientas existentes cumplen su función correctamente. Por ello, basta
con que al buscar un objeto, se desplieguen también resultados de
investigaciones que se hayan realizado al respecto con un enlace a SIMBAD o
similares.

\end{enumerate}
